\RequirePackage{luatex85}
\documentclass[handout]{beamer}
\setbeamertemplate{caption}[numbered]
\usepackage{textpos}
\usepackage{listings}
\usepackage{graphicx}
\usepackage{xcolor}
\usepackage{csquotes}
\usepackage{xcolor}
\definecolor{mygreen}{rgb}{0,0.6,0}
\definecolor{mygray}{rgb}{0.5,0.5,0.5}

\newcommand\Wider[2][3em]{%
\makebox[\linewidth][c]{%
  \begin{minipage}{\dimexpr\textwidth+#1\relax}
  \raggedright#2
  \end{minipage}%
  }%
}

\lstset{language=C++,
           basicstyle=\ttfamily\scriptsize,
           keywordstyle=\color{blue}\ttfamily,
           stringstyle=\color{red}\ttfamily,
           commentstyle=\color{mygreen}\ttfamily,
          breaklines=true,
          captionpos=b,
          numbers=left,
          numbersep=5pt,
          numberstyle=\tiny\color{mygray},
          rulecolor=\color{black},
          xleftmargin=\parindent,
          frame=single,
          backgroundcolor=\color{white}
}

\setbeamercolor{normal text}{fg=black,bg=white}
\definecolor{beamer@blendedblue}{rgb}{0,0,0}
\setbeamercolor{structure}{fg=beamer@blendedblue}


\title{Introduction to Object Detection with Convolution Neural Networks using the TensorFlow API}
\author{
	\includegraphics[width=3cm]{../../media/logo/NVLogo_2D.eps}
	\vspace{0.75cm}
	\\Abel Brown}
\date{\today}

\begin{document}

\frame{\titlepage}

\begin{frame}{Outline}
\tableofcontents
\end{frame}


\addtobeamertemplate{frametitle}{}{%
\begin{textblock*}{200mm}(.75\textwidth,-0.35cm)
\includegraphics[width=3cm]{../../media/logo/NVLogo_2D_H.eps}
\end{textblock*}}

\addtobeamertemplate{navigation symbols}{}{%
    \usebeamerfont{footline}%
    \usebeamercolor[fg]{footline}%
    \hspace{1em}%
    \insertframenumber/\inserttotalframenumber
}

\section{What this lab is}
\begin{frame}{What this lab is}
\begin{itemize}
\itemsep 1em
	\item<>Get going fast!  
	\item<>Leverage the Google Object Detection API in TensorFlow
	\item<>Gain experience using state-of-the-art object detection network architectures and understand engineering trade-offs
	\item<>Gain familiarity with the Microsoft COCO dataset
	\item<>Build intuition and codes for evaluating detector performance
\end{itemize}
\end{frame}

\section{What this lab is not}
\begin{frame}{What this lab is not}
\begin{itemize}
\itemsep 1em
	\item<>Not an intro to machine learning from first principles
	\item<>Not a rigorous formalism of convolutional neural networks
	\item<>Not a survey of all the features and options of TensorFlow
\end{itemize}
\end{frame}


\section{What we hope you take away}
\begin{frame}{Key Takeaways }
\begin{itemize}
\itemsep 1em
	\item<>Baseline for state-of-the-art object detection w/ deep learning
	\item<>Understanding of the complexities in evaluating performance
	\item<>Familiarity with industry standard COCO dataset
	\item<>Everything you need to start leveraging top performing deep learning based object detection models in your own work
\end{itemize}
\end{frame}

\section{Object Detection API in TensorFlow}
\begin{frame}{Object Detection API in TensorFlow}
\begin{itemize}
\itemsep 1em
	\item<>The TensorFlow Object Detection API is an open source framework built on top of TensorFlow
	\item<>This API makes it easy to construct, train and deploy object detection models
	\item<>The API includes a selection of trainable detection models:
	\begin{itemize}\setbeamertemplate{itemize items}[square]
		\item<>Single Shot Multibox Detector (SSD) with MobileNets
		\item<>SSD with Inception V2
		\item<>Region-Based Fully Convolutional Networks (R-FCN) with Resnet 101
		\item<>Faster RCNN with Resnet 101
		\item<>Faster RCNN with Inception Resnet v2
	\end{itemize}
	\item<>All models have been pre-trained on the COCO dataset and are ready for use out-of-the-box
	\item<>Convenient local training scripts included as well
\end{itemize}
\end{frame}


\begin{frame}{Object Detection API in TensorFlow}
\begin{itemize}
\itemsep 1em
	\item<>The SSD models that use MobileNet are lightweight, so that they can be comfortably run in real time on mobile devices
	\item<>The Faster RCNN models are much more computationally intensive but are significantly more accurate	
\end{itemize}
\end{frame}


\section{The MSCOCO dataset}
\begin{frame}{The Microsoft COCO Dataset}
\begin{itemize}
\itemsep 1em
	\item<>In this lab we're going to use the Common Objects in Context (COCO) dataset from Microsoft .
	\item<>COCO is a new image recognition, segmentation, and captioning dataset. COCO has several features:
	\begin{itemize}\setbeamertemplate{itemize items}[square]
		\item<>Object segmentation
		\item<>Recognition in Context
		\item<>Multiple objects per image
		\item<>More than 300,000 images
		\item<>More than 2 Million instances
		\item<>80 object categories
		\item<>5 captions per image
		\item<>Keypoints on 100,000 people
	\end{itemize}
\end{itemize}
\end{frame}


\section{Launching the Lab}
\begin{frame}{Lets Get Started \ldots}
\begin{itemize}
\itemsep 1em
	\item<>Navigate to \textcolor{blue}{nvlabs.qwiklab.com}\footnote{Chrome browser usually works best}
	\item<>Login or create a new account
	\item<>Select the \enquote{DLI Instructor-Led Labs? Class}
	\item<>Find the lab called \enquote{Introduction to Object Detection with TensorFlow}, select it, click Select, and finally click Start
	\item<>When prompted, enter your 16-digit token code
	\item<>After a short wait, lab Connection information will be shown
	\item<>Please ask Lab Assistants for help!
\end{itemize}
\end{frame}


\end{document}
